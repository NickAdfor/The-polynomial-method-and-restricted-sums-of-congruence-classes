\documentclass[11pt]{ctexart}
\usepackage{geometry}
\usepackage{amsmath,amsthm,amssymb}
\usepackage[table]{xcolor}
\usepackage{fullpage}
\usepackage{enumitem}
\usepackage{tikz-cd}
\usepackage{setspace}
\usepackage{etoolbox}

\setlength{\parindent}{0.0in}
\setlength{\parskip}{12pt}



\def\sep{\vspace{1cm}\hrule\vspace{1cm}}
\def\tab{\;\;\;\;\;\;}

%%%%%%%%%%%%%%%%%%%%%%%%%%%%%%%%%%%%%%%%%%%%%

\def\R{\mathbb{R}}
\def\Q{\mathbb{Q}}
\def\N{\mathbb{N}}
\def\C{\mathbb{C}}
\def\Z{\mathbb{Z}}

\def\a{\alpha}
\def\b{\beta}
\def\c{\gamma}
\def\d{\delta}
\def\e{\epsilon}
\def\h{\theta}
\def\w{\omega}

\def\bu{\mathbf{u}}
\def\bv{\mathbf{v}}

\def\iff{\Longleftrightarrow}
\def\to{\rightarrow}
\def\inj{\hookrightarrow}
\def\surj{\twoheadrightarrow}

\def\x{\times}
\def\<{\langle}
\def\>{\rangle}
\def\oo{\infty}
\def\normal{\triangleleft}

\renewcommand{\vec}[1]{\overrightarrow{#1}}
\renewcommand{\bf}[1]{\mathbf{#1}}

\newcommand{\mat}[1]{\begin{pmatrix}#1 \\ \end{pmatrix}}
\newcommand{\case}[2][l]{\left\{\begin{array}{#1}#2 \\ \end{array}\right.}


%%%%%%%%%%%%%%%%%%%%%%%%%%%%%%%%%%%%%%%%%%%%% %定义solutions 环境
\newlist{solutions}{description}{1}
\setlist[solutions]{
	labelwidth=0pt,          % 取消标签宽度
	leftmargin=0pt,          % 取消左边距
	font=\normalfont\bfseries, % 小标题加粗
	itemsep=1\baselineskip,   % 条目间间隔一行
}

\newenvironment{question}{\itshape}{} % 问题用斜体 可以在这个环境里抄写题目并用下划线强调什么的
\newenvironment{solution}{\par\medskip}{} % 解答用正体

\begin{document}

\begin{solutions}
\item[]
\begin{question}

\end{question}

\begin{solution}
    令 $\mathbb{F}_p=\mathbb Z/p\mathbb Z$($p$ 为素数)。

    固定整数 $k\ge 0$。对每个 $i\in\{0,\dots,k\}$ 令 $A_i\subset\mathbb{F}_p$ 且 $|A_i|=c_i+1$。

    设 $h(x_0,\dots,x_k)\in\mathbb{F}_p[x_0,\dots,x_k]$ 为一个多项式。记 $\deg(h)$ 为 $h$ 的总次数。

    设整数 $m\ge 0$ 满足
    \[
    m+\deg(h)=\sum_{i=0}^k c_i.
    \]

    记 $\Sigma(x):=x_0+\cdots+x_k$(这是你代码里的 `sumX`)。

    记向量 $c=(c_0,\dots,c_k)$;我们关心的目标单项式是 $x^c=x_0^{c_0}\cdots x_k^{c_k}$。

    假设给出的代数条件(对应你代码里的 `h_coeff`):
    \[
    [x^c]\bigl(\Sigma^m\cdot h\bigr)\ne 0 \qquad\text{(在 $\mathbb{F}_p$ 中)}.
    \]

    定义受限和集
    \[
    S:=\bigl\{\Sigma(a)=a_0+\cdots+a_k:\ a_i\in A_i,\ h(a)\ne 0\bigr\}.
    \]
  
  我们要证明 $|S|\ge m+1$ 且 $m<p$。
  
  
  
   第一部分:证明 $|S|\ge m+1$
  
  ### 思路概览
  
  若 $|S|\le m$,把 $S$ 包含进一个大小为 $m$ 的多重集合 $E\subset\mathbb{F}_p$,构造多项式
  \[
  Q(x):=h(x)\cdot\prod_{e\in E}\bigl(\Sigma(x)-e\bigr),
  \]
  证明 $Q$ 在 $A_0\times\cdots\times A_k$ 上处处为零,但通过系数分析得到 $Q$ 的某一指定单项式系数非零,从而矛盾。
  
  下面逐条给出证明中每一条 \textbf{Claim} (have)。
  
  
  
  ### \textbf{Claim} 1 (定义 $E$)
  
  断言:若 $|S|\le m$,则存在多重集合(或可以视作大小为 $m$ 的列表) $E\subset \mathbb{F}_p$ 使得 $S\subseteq E$ 并 $|E|=m$。
  
  证明:直接取 $E$ 为在集合 $S$ 的元素后补充若干同一元素(例如 $0$)使得总数为 $m$。这是纯组合的存在性构造。
  
  
  
  ### \textbf{Claim} 2(定义 $Q$)
  
  断言:定义
  \[
  Q(x):=h(x)\cdot\prod_{e\in E}\bigl(\Sigma(x)-e\bigr).
  \]
  这是我们用于构造矛盾的关键多项式。
  
  
  
  ### \textbf{Claim} 3($Q$ 在所有 $A_0\times\cdots\times A_k$ 上为零)
  
  断言:对任意 $x\in A_0\times\cdots\times A_k$,都有 $Q(x)=0$。
  
  证明:

    若 $h(x)=0$,显然 $Q(x)=0$。

    若 $h(x)\ne 0$,则 $\Sigma(x)\in S\subseteq E$,因此 $\prod_{e\in E}(\Sigma(x)-e)=0$。所以 $Q(x)=0$。
  
  
  
  ### \textbf{Claim} 4(总次数:$\deg P = m$ 和 $\deg Q = \sum c_i$)
  
  令
  \[
  P(x):=\prod_{e\in E}(\Sigma(x)-e).
  \]
  断言 (a):$\deg(P)=m$。
  
  证明 (a):每一因子 $\Sigma-e$ 的总次数是 1(因为 $\Sigma$ 的每项都是度 1 且常数项不改变总度),乘积共有 $m$ 因子,所以总次数为 $m$。
  
  断言 (b):因此 $\deg(Q)=\deg(h)+\deg(P)=\deg(h)+m=\sum_{i=0}^k c_i$。
  
  
  
  ### \textbf{Claim} 5(把 $P$ 写成 $\Sigma^m+R$,且 $\deg R< m$)
  
  断言:存在多项式 $R$ 使得
  \[
  P=\Sigma^m+R,
  \quad\text{且}\quad \deg(R)<m.
  \]
  
  证明:将每个因子 $\Sigma-e$ 展开为 $\Sigma - e$ 并把所有乘出来的最高次项合并得 $\Sigma^m$(对应取每个因子中的 $\Sigma$ 项),其余项每次至少丢失一次 $\Sigma$ 的选择,从而总次数严格小于 $m$。
  
  
  
  ### \textbf{Claim} 6(展开 $Q$ 并分析 $[x^c]Q$)
  
  断言:将 $Q=h\cdot P = h\Sigma^m + hR$。我们要计算 $[x^c]Q$。
  
  先看几个子断言。
  
  \textbf{Claim} 6.1 (leading term):$[x^c]P=[x^c]\Sigma^m$。
  
  证明:任何来自 $R$ 的单项 $x^d$ 满足 $\deg(d)<m$。而为了配出 $x^c$(其总度为 $\sum c_i = m+\deg(h)$)在乘 $h$ 后贡献到 $[x^c](hP)$ 中,若 $h$ 取的是常数项之外的某个 $x^d$($\deg(d)>0$),则需要 $P$ 为相应的较高度单项来补齐;但 $R$ 的总度 $< m$ 无法为这些情形提供所需的高次度。更直接地说:在 $P$ 的分解中,只有 $\Sigma^m$ 的高次项可能在乘以 $h$ 后产生总度为 $\sum c_i$ 的项,因此 $[x^c]P=[x^c]\Sigma^m$。
  
  \textbf{Claim} 6.2 (other terms vanish):若 $d$ 是 $h$ 中的一个非零次数向量(即 $d\ne 0$),则
  \[
  [x^{c-d}]P = 0.
  \]
  
  证明:$\deg(c-d)=\sum c_i - \deg(d)= m+\deg(h)-\deg(d) > m$ (当 $\deg(d)>0$ 时),但 $P$ 的最高总度是 $m$,因此该系数为零。
  
  \textbf{Claim} 6.3(结合):因此,在计算 $[x^c]Q=[x^c](hP)$ 时,唯一可能非零的乘积项是当 $h$ 取常数项 $[x^0]h$ 时与 $[x^c]P$ 相乘产生的项。所以
  \[
  [x^c]Q = [x^0]h\cdot [x^c]P = [x^0]h\cdot [x^c]\Sigma^m.
  \]
  
  
  
  ### \textbf{Claim} 7(关于 $h$ 的常数项)
  
  断言:在你的证明框架中通常需要把 $h$ 的常数项当作 1(或至少非零),以保证 $[x^c]Q$ 与 $[x^c]\Sigma^m$ 同非零性相等。我们具体使用的是:$[x^0]h\ne 0$。
  
  说明:你原始的假设给出 $[x^c](\Sigma^m h)\ne 0$。在前面的分解
  \[
  [x^c](\Sigma^m h)=\sum_{d} [x^d]h\cdot [x^{c-d}]\Sigma^m,
  \]
  当 $[x^{c-d}]\Sigma^m$ 只有在 $d=0$ 时可能非零(因为$\deg(\Sigma^m)=m$),可推出 $[x^0]h\cdot [x^c]\Sigma^m\ne 0$。因此 必有 $[x^0]h\ne 0$ 且 $[x^c]\Sigma^m\ne 0$。(换句话说,$[x^0]h$ 至少是非零的;若你想把它标准化为 1,可以把 $h$ 除以该非零常数因子。)
  
  
  
  ### \textbf{Claim} 8(因此 $[x^c]Q\ne 0$)
  
  断言:由上面,$[x^c]Q=[x^0]h\cdot [x^c]\Sigma^m\ne 0$。
  
  证明:由 \textbf{Claim} 6.3 与 \textbf{Claim} 7。
  
  
  
  到此我们得出:

    $Q$ 在每个 $A_0\times\cdots\times A_k$ 的点处取值为 0(\textbf{Claim} 3),

    而 $Q$ 的某个固定单项 $x^c$ 的系数非零(\textbf{Claim} 8)。
  
  接下来我们将通过构造一个把变量次数降低到 $\le c_i$ 的多项式 $Q_{\mathrm{bar}}$,得到同样的矛盾。
  
  
  
  ### \textbf{Claim} 9(定义消元多项式 $g_i$)
  
  对每个 $i$ 定义
  \[
  g_i(x_i):=\prod_{a\in A_i} (x_i-a).
  \]
  显然 $g_i$ 的次数是 $c_i+1$,且对任意 $a\in A_i$, $g_i(a)=0$。
  
  
  
  ### \textbf{Claim} 10(通过替换把 $Q$ 降低至每变量度 $\le c_i$:定义 $Q_{\mathrm{bar}}$)
  
  构造说明:以如下规则反复替换 $Q$ 中任何出现的 $x_i^{c_i+1}$(或更高次):
  用 $g_i(x_i)=x_i^{c_i+1} - (\text{低次项})$ 解出
  \[
  x_i^{c_i+1} = (\text{低次项}) \quad(\text{模 } g_i),
  \]
  把高次项替换为低次项,直到每个变量的出现次数均 $\le c_i$。所得多项式记为 $Q_{\mathrm{bar}}$。
  
  (这是你代码实现 `reduce_polynomial_degrees` 的数学描述。)
  
  
  
  ### \textbf{Claim} 11(替换不改变在 $A_0\times\cdots\times A_k$ 上的值)
  
  断言:对于任意 $x\in A_0\times\cdots\times A_k$,都有 $Q_{\mathrm{bar}}(x)=Q(x)=0$。
  
  证明:每一次替换规则都是用恒等式
  \[
  x_i^{c_i+1} - (\text{低次多项}) = g_i(x_i),
  \]
  对多项式做代换;在点 $x$ 上,如果 $x_i\in A_i$ 则 $g_i(x_i)=0$。因此每一次替换都在这些点上保持多项式取相同的值。既然初始的 $Q(x)=0$,替换后仍为 0。
  
  
  
  ### \textbf{Claim} 12(替换后每个变量的次数受限)
  
  断言:$Q_{\mathrm{bar}}$ 对每个 $i$ 满足 $\deg_{x_i} Q_{\mathrm{bar}} \le c_i$。
  
  证明:替换规则的目的是消除所有指数超过 $c_i$ 的幂,因此最终得到的多项式每个变量次数至多 $c_i$。
  
  
  
  ### \textbf{Claim} 13(应用组合 Nullstellensatz / 逐变量度控制的消失引理)
  
  引理(多变量版):设 $P\in\mathbb{F}_p[x_0,\dots,x_k]$。若对每个 $i$, $\deg_{x_i}P<|A_i|$,且 $P$ 在 $A_0\times\cdots\times A_k$ 的所有点上均取零值,则 $P$ 为零多项式。
  
  应用:由于 $|A_i|=c_i+1$ 且 $\deg_{x_i}Q_{\mathrm{bar}}\le c_i$,我们有 $\deg_{x_i}Q_{\mathrm{bar}}<|A_i|$。再加上 \textbf{Claim} 11(在笛卡尔积上处处为 0),由引理可得:
  
  断言:$Q_{\mathrm{bar}}=0$。
  
  
  
  ### \textbf{Claim} 14(替换过程不改变目标系数)
  
  断言:替换过程中不会改变 $x^c$ 的系数,所以
  \[
  [x^c]Q_{\mathrm{bar}}=[x^c]Q\ne 0.
  \]
  
  证明:在替换过程中,任何一次用 $x_i^{c_i+1}=$(低次项) 替换都会把某个指数分量 $\ge c_i+1$ 的单项替为若干单项,这些新单项在该变量方向的指数都 $\le c_i$。因此目标单项 $x^c$(在每个分量刚好是 $c_i$)不会被“生成”自替换的结果;反过来,也不会被替换为别的项而丢失——更精确地说:替换只会把具有某个变量指数 $>c_i$ 的单项写成若干次幂中指数 $\le c_i$ 的线性组合,但这些替换不会改变原来那一项 $x^c$ 的系数,因为 $x^c$ 本身的每个分量都不超过 $c_i$,即替换不会触及已经恰好等于上限的幂。形式化可以通过追踪替换过程中每一步对目标单项系数的影响来完成(每一步对目标单项系数的改变量均为 0)。
  
  
  
  矛盾:由 \textbf{Claim} 13 我们有 $Q_{\mathrm{bar}}=0$,但由 \textbf{Claim} 14 我们有 $[x^c]Q_{\mathrm{bar}}\ne 0$。矛盾。
  
  因此假设 $|S|\le m$ 必为假的,从而得出
  \[
  \boxed{|S|\ge m+1.}
  \]
  这是第一部分结论。
  
  
  
   第二部分:证明 $m<p$
  
  我们现在证明 $m<p$。思路是:如果 $m\ge p$,利用特征 $p$ 下的 “Freshman's dream” 与指数模 $p$ 的整除性质,结合每个 $c_i \le p-1$(因为 $c_i+1=|A_i|\le p$,集合 $A_i\subset\mathbb{F}_p$ 的大小至多 $p$)得到与前面关于系数非零的断言矛盾。
  
  下面详细写出每一步 have。
  
  
  
  ### \textbf{Claim} 15(每个 $c_i$ 小于 $p$)
  
  断言:对每个 $i$, $0\le c_i \le p-1$。
  
  证明:因为 $A_i\subset\mathbb{F}_p$,所以 $|A_i|\le p$。由 $|A_i|=c_i+1$ 得 $c_i\le p-1$。显然 $c_i\ge 0$。
  
  
  
  ### 假设反面并写 $m=pq+r$
  
  假设 $m\ge p$。将 $m$ 写成 $m=pq+r$ 其中 $q\ge 1$ 且 $0\le r\le p-1$(这是整除带余数分解)。
  
  
  
  ### \textbf{Claim} 16($\Sigma^{p} = \sum_i x_i^{p}$ 在特征 $p$ 下)
  
  断言:
  \[
  (\Sigma)^p = (x_0+\dots+x_k)^p = x_0^p+\dots+x_k^p.
  \]
  
  证明:在特征 $p$ 的域中,二项式(多项式)的中间项都乘以二项式系数 $\binom{p}{j}$($0<j<p$),而这些系数都被 $p$ 整除,故在 $\mathbb{F}_p$ 中为零。因此展开只剩下 $x_i^p$ 项之和。
  
  
  
  ### \textbf{Claim} 17(形式分解 $\Sigma^m = (\Sigma^p)^q \cdot \Sigma^r$)
  
  断言:
  \[
  \Sigma^m = (\Sigma^{p})^q \cdot \Sigma^r = \bigl(\sum_{i=0}^k x_i^p\bigr)^q \cdot \Sigma^r.
  \]
  
  这是代数上直接成立的幂的分解。
  
  
  
  ### \textbf{Claim} 18(每个单项在 $(\sum x_i^p)^q$ 中的指数都能被 $p$ 整除)
  
  断言:任一出现在 $(\sum_i x_i^p)^q$ 展开中的单项形如 $\prod_i x_i^{p\cdot t_i}$(也就是每个变量的指数均为 $p$ 的倍数)。
  
  证明:显然,因为每项来自若干次选取 $x_j^p$ 因子并相乘,指数是 $p$ 的倍数。
  
  
  
  ### \textbf{Claim} 19(若 $c_i<p$ 中某些 $c_i$ 非零,则 $[x^c](\Sigma^{pq})=0$)
  
  断言:由于 $c_i<p$(\textbf{Claim} 15),在 $(\sum x_i^p)^q$ 中不可能出现指数恰好等于 $c=(c_0,\dots,c_k)$(除非 $c=0$),因此 $[x^c](\Sigma^{pq})=0$。
  
  证明:任一单项在 $(\sum x_i^p)^q$ 中每个分量都是 $p$ 的倍数,而 $0\le c_i\le p-1$;若某 $c_i\ne 0$(或任一 $c_i$ 不是 $p$ 的倍数,显然),就无法匹配一个每分量为 $p$-倍数的向量,故系数为 0。注意 $c$ 的总度 $\sum c_i = m+\deg(h) \ge m\ge p$,所以 $c$ 不是全零向量,因此至少有一个分量大于 $0$。
  
  
  
  ### \textbf{Claim} 20(结合 $\Sigma^m = (\Sigma^{pq})\Sigma^r$ 得到 $[x^c]\Sigma^m = 0$)
  
  断言:在上述分解中,$[x^c]\Sigma^m = 0$。
  
  证明:任何单项 $x^c$ 若要在乘积中出现,必须是从某个单项 $x^u$(来自 $(\sum x_i^p)^q$)和某个单项 $x^v$(来自 $\Sigma^r$)相乘得到,且 $u+v=c$。但 $u$ 的每个分量是 $p$ 的倍数,而 $v$ 的总度是 $r < p$。因此 $u$ 的总度至少是 0 并且是 $p$ 的倍数;而 $\sum c_i = \sum u_i + \sum v_i$。由于 $\sum v_i=r<p$ 且 $\sum c_i\ge m\ge p$,必须有 $\sum u_i\ge p$。但 $\sum u_i$ 是 $p$ 的倍数,因此至少为 $p$。若要满足每个分量 $u_i$ 都为 $p$ 倍,而 $c_i<p$(每个分量 $< p$),那意味着 $u_i$ 必须全部为 $0$。 然而若所有 $u_i=0$,则 $v=c$ 并且 $\sum v_i=\sum c_i \ge p$,这与 $\sum v_i=r<p$ 矛盾。因此不存在这样的分解,故 $[x^c]\Sigma^m=0$。
  
  (这一点需要细心检验:核心思想是指数的模 $p$ 性质与总度大小的不兼容;因为 $u$ 每个分量被 $p$ 整除而 $c$ 每个分量都在 $0,\dots,p-1$ 之间,矛盾产生。)
  
  
  
  ### \textbf{Claim} 21(与先前关于 $[x^c](\Sigma^m h)\ne 0$ 矛盾)
  
  断言:若 $[x^c]\Sigma^m = 0$,则对所有 $d$ 有 $[x^{c-d}]\Sigma^m = 0$(因为 $\deg(\Sigma^m)=m$,可用同类理由),从而
  \[
  [x^c](\Sigma^m h) = \sum_d [x^d]h\cdot [x^{c-d}]\Sigma^m = 0,
  \]
  与初始假设 $[x^c](\Sigma^m h)\ne 0$ 矛盾。
  
  
  
  因此假设 $m\ge p$ 导致矛盾,故必有
  \[
  \boxed{m<p.}
  \]
  
  
  
   结论
  
  综合两部分结果,我们得出在给定假设下(特别是假设 $[x^c](\Sigma^m h)\ne 0$ 且 $|A_i|=c_i+1$):

    $|S| \ge m+1$,

    $m < p$。
  
  这正是你代码欲证明的结论(ANR 多项式方法的要点)。
  
  
  
   进一步说明与可选证明细节
  
  1. 关于 \textbf{Claim} 6 的严格性:上面对“其他项为 0” 的论证用了“总次数的比较”这一直观但严谨的方式;形式化时可以以“如果 $h$ 取非零次数向量 $d\ne 0$ 则 $\deg(c-d)>\deg(P)=m$ ”为线索,逐项说明相应系数在 $P$ 中为 0。
  
  2. 关于替换不改变目标系数(\textbf{Claim} 14):形式化证明可以用归纳:考虑一个单次替换 $x_i^{t}\mapsto x_i^{t-(c_i+1)}\cdot g_i(x_i)$ 的反写,追踪目标单项在替换前后系数的变化,证明不会影响恰好指数为 $c$ 的项。或者更简洁地把替换视为在多项式环里把 $Q$ 映射到商环
     \[
     \mathbb{F}_p[x_0,\dots,x_k]/(g_0,\dots,g_k)
     \]
     中的代表;在该商环里,每个 $x_i^{c_i+1}$ 被替换为低次多项,从而 $Q$ 在代表类中与 $Q_{\mathrm{bar}}$ 一致,而 $x^c$ 为该商环中仍然代表一个非零基向量(因为 $c_i \le p-1$ 并且 $x^c$ 的系数在商环中被保留),所以在系数层面不变。
  
  3. 关于第二部分($m<p$)的关键点:要点是 $c_i<p$(每个分量小于 $p$),而在 $(\sum x_i^p)^q$ 中出现的所有指数分量都是 $p$ 的倍数,因此不可能拼凑出每分量都位于 $0,\dots,p-1$ 的 $c$。这类模 $p$ 的指数-余数考虑是特征 $p$ 微妙但强有力的工具。
  
\end{solution}
\end{solutions}
\end{document}