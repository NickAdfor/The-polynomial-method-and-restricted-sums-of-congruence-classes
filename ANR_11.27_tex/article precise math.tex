\documentclass[11pt]{ctexart}
\usepackage{geometry}
\usepackage{amsmath,amsthm,amssymb}
\usepackage[table]{xcolor}
\usepackage{fullpage}
\usepackage{enumitem}
\usepackage{tikz-cd}
\usepackage{setspace}
\usepackage{etoolbox}

\setlength{\parindent}{0.0in}
\setlength{\parskip}{12pt}



\def\sep{\vspace{1cm}\hrule\vspace{1cm}}
\def\tab{\;\;\;\;\;\;}

%%%%%%%%%%%%%%%%%%%%%%%%%%%%%%%%%%%%%%%%%%%%%

\def\R{\mathbb{R}}
\def\Q{\mathbb{Q}}
\def\N{\mathbb{N}}
\def\C{\mathbb{C}}
\def\Z{\mathbb{Z}}

\def\a{\alpha}
\def\b{\beta}
\def\c{\gamma}
\def\d{\delta}
\def\e{\epsilon}
\def\h{\theta}
\def\w{\omega}

\def\bu{\mathbf{u}}
\def\bv{\mathbf{v}}

\def\iff{\Longleftrightarrow}
\def\to{\rightarrow}
\def\inj{\hookrightarrow}
\def\surj{\twoheadrightarrow}

\def\x{\times}
\def\<{\langle}
\def\>{\rangle}
\def\oo{\infty}
\def\normal{\triangleleft}

\renewcommand{\vec}[1]{\overrightarrow{#1}}
\renewcommand{\bf}[1]{\mathbf{#1}}

\newcommand{\mat}[1]{\begin{pmatrix}#1 \\ \end{pmatrix}}
\newcommand{\case}[2][l]{\left\{\begin{array}{#1}#2 \\ \end{array}\right.}


%%%%%%%%%%%%%%%%%%%%%%%%%%%%%%%%%%%%%%%%%%%%% %定义solutions 环境
\newlist{solutions}{description}{1}
\setlist[solutions]{
	labelwidth=0pt,          % 取消标签宽度
	leftmargin=0pt,          % 取消左边距
	font=\normalfont\bfseries, % 小标题加粗
	itemsep=1\baselineskip,   % 条目间间隔一行
}

\newenvironment{question}{\itshape}{} % 问题用斜体 可以在这个环境里抄写题目并用下划线强调什么的
\newenvironment{solution}{\par\medskip}{} % 解答用正体

\begin{document}

\begin{solutions}
\item[]
\begin{question}

\end{question}

\begin{solution}
    Definition 1 (Elimination Polynomials). Let \(p\) be a prime, \(k \in \mathbb{N}\), and \(A : \mathrm{Fin}(k+1) \to \mathcal{F}(\mathbb{Z}/p\mathbb{Z})\). We define the elimination polynomials as:
    \[
    g_i(X) = \prod_{a \in A_i} (X_i - a), \quad \forall i \in \mathrm{Fin}(k+1)
    \]
    
    Definition 2 (Degree Reduction of a Polynomial). Let \(P \in \mathrm{MvPoly}(\mathrm{Fin}(k+1), \mathbb{Z}/p\mathbb{Z})\), \(g : \mathrm{Fin}(k+1) \to \mathrm{MvPoly}(\mathrm{Fin}(k+1), \mathbb{Z}/p\mathbb{Z})\), and \(c : \mathrm{Fin}(k+1) \to \mathbb{N}\). We define the reduced polynomial as:
    \[
    \mathrm{reduce\_polynomial\_degrees}(P,g,c) = \sum_{m \in \mathrm{support}(P)} \begin{cases}
    \mathrm{coeff}_m(P) \cdot \mathrm{monomial}_{m'} 1 \cdot g_i & \text{if } \exists i, m_i > c_i \\
    \mathrm{coeff}_m(P) \cdot \mathrm{monomial}_m 1 & \text{otherwise}
    \end{cases}
    \]
    where \(m' = \mathrm{update}(m, i, m_i - (c_i + 1))\).
    
    Lemma 2.2 (Vanishing on a Product Finset). Let \(\sigma\) be a finite type, \(R\) an integral domain, \(P \in \mathrm{MvPoly}(\sigma, R)\), and \(S : \sigma \to \mathcal{F}(R)\) such that:
    
    1. \(\forall i, \mathrm{degreeOf}_i(P) < |S(i)|\)
    
    2. \(\forall x : \sigma \to R, (\forall i, x_i \in S_i) \Rightarrow \mathrm{eval}_x P = 0\)
    
    Then \(P = 0\).

Theorem 2.1 (Alon--Nathanson--Ruzsa). Let \(p\) be prime, \(k \in \mathbb{N}\), and assume:

  \(h \in \mathrm{MvPoly}(\mathrm{Fin}(k+1), \mathbb{Z}/p\mathbb{Z})\)

  \(A : \mathrm{Fin}(k+1) \to \mathcal{F}(\mathbb{Z}/p\mathbb{Z})\)

  \(c : \mathrm{Fin}(k+1) \to \mathbb{N}\)

  \(\forall i, |A_i| = c_i + 1\)

  \(m = (\sum_i c_i) - \mathrm{totalDegree}(h)\)

  \(\mathrm{coeff}_c\left((\sum_i X_i)^m \cdot h\right) \neq 0\)

Define the restricted sumset:
\[
S = \left\{ \sum_i f_i : f \in \prod_i A_i, h(f) \neq 0 \right\}
\]
Then:

1. \(|S| \geq m + 1\)

2. \(m < p\)d

Proof.

Part 1: Proof that \(|S| \geq m + 1\)

Assume for contradiction that \(|S| \leq m\).

Step 1: Construction of auxiliary set \(E\)
Since \(|S| \leq m\), there exists a multiset \(E\) with \(S \subseteq E\) and \(|E| = m\). Formally, we take \(E = S \cup \{\text{zeros}\}\) where we add enough zeros to make the cardinality exactly \(m\).

Step 2: Polynomial construction and vanishing property
Define \(\mathrm{sumX} = \sum_{i} X_i\) and \(Q = h \cdot \prod_{e \in E} (\mathrm{sumX} - C e)\).

We prove \(Q\) vanishes on \(\prod_i A_i\): For any \(x \in \prod_i A_i\), if \(h(x) \neq 0\) then \(\sum_i x_i \in S \subseteq E\), so one factor \((\mathrm{sumX} - C(\sum_i x_i))\) in the product vanishes at \(x\); if \(h(x) = 0\) then clearly \(Q(x) = 0\).

Step 3: Total degree analysis of \(Q\) - THE TECHNICAL HEART OF THE PROOF

We now analyze the total degree of the fundamental building block polynomials.

Theorem: For any \(e \in \mathbb{Z}/p\mathbb{Z}\), \(\mathrm{totalDegree}(\sum_i X_i - C e) = 1\).

Proof of upper bound: \(\mathrm{totalDegree}(\sum_i X_i - C e) \leq 1\)

We examine the explicit form of the polynomial:
\[
P = \sum_i X_i - C e = X_0 + X_1 + \cdots + X_k - e
\]

This is a sum of:

  \(k+1\) terms of the form \(X_i\), each being a monomial of total degree 1

  One constant term \(-e\), which is a monomial of total degree 0

The total degree of a polynomial is defined as the maximum total degree among all monomials with nonzero coefficients in its support. Since all monomials in \(P\) have total degree \(\leq 1\), we immediately conclude that \(\mathrm{totalDegree}(P) \leq 1\).

Proof of lower bound: \(\mathrm{totalDegree}(\sum_i X_i - C e) \geq 1\)

To prove this, we must demonstrate the existence of at least one monomial of total degree 1 that has a nonzero coefficient in \(P\).

Consider any specific variable \(X_i\). Examine the monomial \(m_i\) that consists solely of \(X_i\) with exponent 1 (and all other variables with exponent 0). The coefficient of this monomial \(m_i\) in \(P\) is:

\[
\mathrm{coeff}_{m_i}(P) = \mathrm{coeff}_{m_i}(\sum_j X_j) - \mathrm{coeff}_{m_i}(C e)
\]

Now analyze each term:

  \(\mathrm{coeff}_{m_i}(\sum_j X_j) = 1\), since among all the \(X_j\) terms, only \(X_i\) contributes to this monomial

  \(\mathrm{coeff}_{m_i}(C e) = 0\), because the constant polynomial \(C e\) contains no monomials of positive degree

Therefore, \(\mathrm{coeff}_{m_i}(P) = 1 - 0 = 1 \neq 0\).

This proves that the monomial \(m_i\) has total degree 1 and appears in \(P\) with nonzero coefficient. Hence, \(\mathrm{totalDegree}(P) \geq 1\).

Combining both inequalities, we conclude that \(\mathrm{totalDegree}(\sum_i X_i - C e) = 1\).

Step 4: Support structure analysis

Now we analyze the support structure of \(P = \sum_i X_i - C e\).

Theorem: The support of \(P\) is contained in \((\bigcup_i \{\mathrm{single}_i 1\}) \cup \{0\}\).

Let \(d\) be any monomial in the support of \(P\), meaning \(\mathrm{coeff}_d(P) \neq 0\). We have:

\[
\mathrm{coeff}_d(P) = \mathrm{coeff}_d(\sum_i X_i) - \mathrm{coeff}_d(C e) \neq 0
\]

We consider cases based on the form of \(d\):

Case 1: \(d = \mathrm{single}_i 1\) for some \(i\) (a monomial consisting of exactly \(X_i^1\))
Then \(\mathrm{coeff}_d(\sum_j X_j) = 1\) and \(\mathrm{coeff}_d(C e) = 0\), so \(\mathrm{coeff}_d(P) = 1 \neq 0\).
Thus \(d \in \bigcup_i \{\mathrm{single}_i 1\}\).

Case 2: \(d = 0\) (the zero monomial, constant term)
Then \(\mathrm{coeff}_d(\sum_j X_j) = 0\) and \(\mathrm{coeff}_d(C e) = e\), so \(\mathrm{coeff}_d(P) = -e\).
If \(e \neq 0\), then \(\mathrm{coeff}_d(P) \neq 0\), so \(d \in \{0\}\).

Case 3: \(d\) has any other form
If \(d\) has total degree \(\geq 2\), then \(\mathrm{coeff}_d(\sum_j X_j) = 0\) and \(\mathrm{coeff}_d(C e) = 0\), so \(\mathrm{coeff}_d(P) = 0\).
If \(d\) has total degree 1 but is not of the form \(\mathrm{single}_i 1\), then it must involve at least two different variables with positive exponents, so again \(\mathrm{coeff}_d(\sum_j X_j) = 0\) and \(\mathrm{coeff}_d(C e) = 0\).

Therefore, the only monomials with nonzero coefficients in \(P\) are those in \((\bigcup_i \{\mathrm{single}_i 1\}) \cup \{0\}\).

Step 5: Degree reduction construction
Construct the elimination polynomials \(g_i = \prod_{a \in A_i} (X_i - C a)\) and define \(\bar{Q} = \mathrm{reduce\_polynomial\_degrees}(Q, g, c)\).

Step 6: Properties of the reduced polynomial
The reduced polynomial \(\bar{Q}\) preserves the vanishing property on \(\prod_i A_i\), satisfies \(\mathrm{degreeOf}_{X_i}(\bar{Q}) \leq c_i\) for all \(i\), and retains the nonzero coefficient of the target monomial.

Step 7: Application of Lemma 2.2 and contradiction
Since:

  \(\bar{Q}\) vanishes on \(\prod_i A_i\)

  \(\mathrm{degreeOf}_{X_i}(\bar{Q}) \leq c_i < |A_i|\) for all \(i\) (since \(|A_i| = c_i + 1\))

By Lemma 2.2, \(\bar{Q} = 0\). But the coefficient of \(\prod_i X_i^{c_i}\) in \(\bar{Q}\) is nonzero, contradiction.

Part 2: Proof that \(m < p\)
Assume \(m \geq p\) for contradiction. In characteristic \(p\), we have the Frobenius identity \((\sum_i X_i)^p = \sum_i X_i^p\). 

When \(m \geq p\), we can write \((\sum_i X_i)^m\) using the Frobenius map, which redistributes the degrees in a way that makes it impossible for the coefficient of \(\prod_i X_i^{c_i}\) in \((\sum_i X_i)^m \cdot h\) to be nonzero, contradicting the hypothesis.

This completes the proof of Theorem 2.1.

\end{solution}
\end{solutions}
\end{document}